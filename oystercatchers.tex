\documentclass[]{article}
\usepackage{lmodern}
\usepackage{amssymb,amsmath}
\usepackage{ifxetex,ifluatex}
\usepackage{fixltx2e} % provides \textsubscript
\ifnum 0\ifxetex 1\fi\ifluatex 1\fi=0 % if pdftex
  \usepackage[T1]{fontenc}
  \usepackage[utf8]{inputenc}
\else % if luatex or xelatex
  \ifxetex
    \usepackage{mathspec}
  \else
    \usepackage{fontspec}
  \fi
  \defaultfontfeatures{Ligatures=TeX,Scale=MatchLowercase}
\fi
% use upquote if available, for straight quotes in verbatim environments
\IfFileExists{upquote.sty}{\usepackage{upquote}}{}
% use microtype if available
\IfFileExists{microtype.sty}{%
\usepackage{microtype}
\UseMicrotypeSet[protrusion]{basicmath} % disable protrusion for tt fonts
}{}
\usepackage[margin=1in]{geometry}
\usepackage{hyperref}
\hypersetup{unicode=true,
            pdftitle={BIO8068 Data visualisation in ecology},
            pdfborder={0 0 0},
            breaklinks=true}
\urlstyle{same}  % don't use monospace font for urls
\usepackage{color}
\usepackage{fancyvrb}
\newcommand{\VerbBar}{|}
\newcommand{\VERB}{\Verb[commandchars=\\\{\}]}
\DefineVerbatimEnvironment{Highlighting}{Verbatim}{commandchars=\\\{\}}
% Add ',fontsize=\small' for more characters per line
\usepackage{framed}
\definecolor{shadecolor}{RGB}{248,248,248}
\newenvironment{Shaded}{\begin{snugshade}}{\end{snugshade}}
\newcommand{\KeywordTok}[1]{\textcolor[rgb]{0.13,0.29,0.53}{\textbf{#1}}}
\newcommand{\DataTypeTok}[1]{\textcolor[rgb]{0.13,0.29,0.53}{#1}}
\newcommand{\DecValTok}[1]{\textcolor[rgb]{0.00,0.00,0.81}{#1}}
\newcommand{\BaseNTok}[1]{\textcolor[rgb]{0.00,0.00,0.81}{#1}}
\newcommand{\FloatTok}[1]{\textcolor[rgb]{0.00,0.00,0.81}{#1}}
\newcommand{\ConstantTok}[1]{\textcolor[rgb]{0.00,0.00,0.00}{#1}}
\newcommand{\CharTok}[1]{\textcolor[rgb]{0.31,0.60,0.02}{#1}}
\newcommand{\SpecialCharTok}[1]{\textcolor[rgb]{0.00,0.00,0.00}{#1}}
\newcommand{\StringTok}[1]{\textcolor[rgb]{0.31,0.60,0.02}{#1}}
\newcommand{\VerbatimStringTok}[1]{\textcolor[rgb]{0.31,0.60,0.02}{#1}}
\newcommand{\SpecialStringTok}[1]{\textcolor[rgb]{0.31,0.60,0.02}{#1}}
\newcommand{\ImportTok}[1]{#1}
\newcommand{\CommentTok}[1]{\textcolor[rgb]{0.56,0.35,0.01}{\textit{#1}}}
\newcommand{\DocumentationTok}[1]{\textcolor[rgb]{0.56,0.35,0.01}{\textbf{\textit{#1}}}}
\newcommand{\AnnotationTok}[1]{\textcolor[rgb]{0.56,0.35,0.01}{\textbf{\textit{#1}}}}
\newcommand{\CommentVarTok}[1]{\textcolor[rgb]{0.56,0.35,0.01}{\textbf{\textit{#1}}}}
\newcommand{\OtherTok}[1]{\textcolor[rgb]{0.56,0.35,0.01}{#1}}
\newcommand{\FunctionTok}[1]{\textcolor[rgb]{0.00,0.00,0.00}{#1}}
\newcommand{\VariableTok}[1]{\textcolor[rgb]{0.00,0.00,0.00}{#1}}
\newcommand{\ControlFlowTok}[1]{\textcolor[rgb]{0.13,0.29,0.53}{\textbf{#1}}}
\newcommand{\OperatorTok}[1]{\textcolor[rgb]{0.81,0.36,0.00}{\textbf{#1}}}
\newcommand{\BuiltInTok}[1]{#1}
\newcommand{\ExtensionTok}[1]{#1}
\newcommand{\PreprocessorTok}[1]{\textcolor[rgb]{0.56,0.35,0.01}{\textit{#1}}}
\newcommand{\AttributeTok}[1]{\textcolor[rgb]{0.77,0.63,0.00}{#1}}
\newcommand{\RegionMarkerTok}[1]{#1}
\newcommand{\InformationTok}[1]{\textcolor[rgb]{0.56,0.35,0.01}{\textbf{\textit{#1}}}}
\newcommand{\WarningTok}[1]{\textcolor[rgb]{0.56,0.35,0.01}{\textbf{\textit{#1}}}}
\newcommand{\AlertTok}[1]{\textcolor[rgb]{0.94,0.16,0.16}{#1}}
\newcommand{\ErrorTok}[1]{\textcolor[rgb]{0.64,0.00,0.00}{\textbf{#1}}}
\newcommand{\NormalTok}[1]{#1}
\usepackage{graphicx,grffile}
\makeatletter
\def\maxwidth{\ifdim\Gin@nat@width>\linewidth\linewidth\else\Gin@nat@width\fi}
\def\maxheight{\ifdim\Gin@nat@height>\textheight\textheight\else\Gin@nat@height\fi}
\makeatother
% Scale images if necessary, so that they will not overflow the page
% margins by default, and it is still possible to overwrite the defaults
% using explicit options in \includegraphics[width, height, ...]{}
\setkeys{Gin}{width=\maxwidth,height=\maxheight,keepaspectratio}
\IfFileExists{parskip.sty}{%
\usepackage{parskip}
}{% else
\setlength{\parindent}{0pt}
\setlength{\parskip}{6pt plus 2pt minus 1pt}
}
\setlength{\emergencystretch}{3em}  % prevent overfull lines
\providecommand{\tightlist}{%
  \setlength{\itemsep}{0pt}\setlength{\parskip}{0pt}}
\setcounter{secnumdepth}{0}
% Redefines (sub)paragraphs to behave more like sections
\ifx\paragraph\undefined\else
\let\oldparagraph\paragraph
\renewcommand{\paragraph}[1]{\oldparagraph{#1}\mbox{}}
\fi
\ifx\subparagraph\undefined\else
\let\oldsubparagraph\subparagraph
\renewcommand{\subparagraph}[1]{\oldsubparagraph{#1}\mbox{}}
\fi

%%% Use protect on footnotes to avoid problems with footnotes in titles
\let\rmarkdownfootnote\footnote%
\def\footnote{\protect\rmarkdownfootnote}

%%% Change title format to be more compact
\usepackage{titling}

% Create subtitle command for use in maketitle
\newcommand{\subtitle}[1]{
  \posttitle{
    \begin{center}\large#1\end{center}
    }
}

\setlength{\droptitle}{-2em}

  \title{BIO8068 Data visualisation in ecology}
    \pretitle{\vspace{\droptitle}\centering\huge}
  \posttitle{\par}
  \subtitle{Further use of ggplot and model interpretation}
  \author{}
    \preauthor{}\postauthor{}
    \date{}
    \predate{}\postdate{}
  

\begin{document}
\maketitle

\subsection{1. Introduction}\label{introduction}

Often you will receive complex data-sets, but initial analyses can be
confusing, and it may require careful interpretation of model outputs to
understand what is the problem. This practical will use a real
ecological dataset to illustrate some of the problems. You can learn
about fine-tuning and polishing ggplot2 graphics in numerous books and
websites, and I particularly recommend The R Cookbook
\url{http://www.cookbook-r.com/Graphs/} which contains the same text as
the associated book. The book R for Data Science by Hadley Wickham
(author of ggplot2 etc.) and its website
\url{https://r4ds.had.co.nz/index.html} are also excellent sources. The
aims of this practical are to:

\begin{itemize}
\tightlist
\item
  show you how to explore ecological data, with common mistakes
\item
  use diagnostic plots to gain better insights
\end{itemize}

\subsection{2. The data}\label{the-data}

The data are from a 3-year study into American oystercatchers,
\emph{Haematopus palliatus}, inhabitating coastal areas near Buenos
Aires, Argentina. Oystercatchers establish nesting territories along the
shoreline, of about 50 to 500 metres in size, and when chicks are being
reared these are defended by adults, with the parents chasing away other
oystercatchers. We will look at a sub-set of the data, for two months,
December and January, when the birds are breeding (southern hemispher
summer).

Oystercatchers use two techniques to break open clam shells, either a
hammering technique or stabbing method. One question in the study was
whether the shells eaten by hammerers are larger than those eaten by
stabbers. Time of year, and location may also affect what is happening,
so we could be looking at a complex 3-way interaction between feeding
type (stabber/hammerer), feeding plot and month.

\subsection{3. Import the data and initial
inspection}\label{import-the-data-and-initial-inspection}

Download the file ``OystercatcherData.txt'' from Blackboard, create a
new project for your oystercatcher data, and within the project create a
subfolder called ``data'' in which to store the downloaded data file.
Create an R script, and import the oystercatcher data into a tibble OC.
As this is tab-separated text format (readable in Excel on Windows),
we'll use read\_tsv rather than read\_csv:

\begin{Shaded}
\begin{Highlighting}[]
\KeywordTok{library}\NormalTok{(readr)}
\NormalTok{OC <-}\StringTok{ }\KeywordTok{read_tsv}\NormalTok{(}\StringTok{"data/OystercatcherData.txt"}\NormalTok{)}
\end{Highlighting}
\end{Shaded}

\begin{verbatim}
## Parsed with column specification:
## cols(
##   ShellLength = col_double(),
##   Month = col_character(),
##   FeedingType = col_character(),
##   FeedingPlot = col_character()
## )
\end{verbatim}

\begin{Shaded}
\begin{Highlighting}[]
\KeywordTok{summary}\NormalTok{(OC)}
\end{Highlighting}
\end{Shaded}

\begin{verbatim}
##   ShellLength       Month           FeedingType        FeedingPlot       
##  Min.   :1.080   Length:197         Length:197         Length:197        
##  1st Qu.:1.850   Class :character   Class :character   Class :character  
##  Median :2.110   Mode  :character   Mode  :character   Mode  :character  
##  Mean   :2.093                                                           
##  3rd Qu.:2.290                                                           
##  Max.   :3.600
\end{verbatim}

\begin{Shaded}
\begin{Highlighting}[]
\CommentTok{# Set the Month, FeedingType and FeedingPlot as factors}
\NormalTok{OC}\OperatorTok{$}\NormalTok{Month <-}\StringTok{ }\KeywordTok{as.factor}\NormalTok{(OC}\OperatorTok{$}\NormalTok{Month)}
\NormalTok{OC}\OperatorTok{$}\NormalTok{FeedingType <-}\StringTok{ }\KeywordTok{as.factor}\NormalTok{(OC}\OperatorTok{$}\NormalTok{FeedingType)}
\NormalTok{OC}\OperatorTok{$}\NormalTok{FeedingPlot <-}\StringTok{ }\KeywordTok{as.factor}\NormalTok{(OC}\OperatorTok{$}\NormalTok{FeedingPlot)}
\KeywordTok{summary}\NormalTok{(OC)}
\end{Highlighting}
\end{Shaded}

\begin{verbatim}
##   ShellLength    Month        FeedingType  FeedingPlot
##  Min.   :1.080   Dec: 79   Hammerers:165   A:66       
##  1st Qu.:1.850   Jan:118   Stabbers : 32   B:53       
##  Median :2.110                             C:78       
##  Mean   :2.093                                        
##  3rd Qu.:2.290                                        
##  Max.   :3.600
\end{verbatim}

\subsubsection{Initial boxplots}\label{initial-boxplots}

A good initial starting point is to produce some boxplots to explore the
data. Using ggplot2, see if you can produce some boxplots similar to
this, for each of the three factors. Store the plots in three ggplot
objects p1, p2 and p3.

\includegraphics{oystercatchers_files/figure-latex/boxplots-1.pdf}

It would be useful to display all three ggplot graphs in one plot,
rather than three separate ones. Go to the R Graphics Cookbook website
\url{http://www.cookbook-r.com/Graphs/Multiple_graphs_on_one_page_(ggplot2)/}
or search on Google for ``R Graphics Cookbook multiplot''. You will see
the page has a simple(?!) R function called ``multiplot'', so copy this
into your R script and run it to make it available. Note that multiplot
uses the ``grid'' package so check that it is installed. Then all you
need is:

\begin{Shaded}
\begin{Highlighting}[]
\KeywordTok{multiplot}\NormalTok{(p1, p2, p3, }\DataTypeTok{cols=}\DecValTok{2}\NormalTok{)}
\end{Highlighting}
\end{Shaded}

\includegraphics{oystercatchers_files/figure-latex/display multiplot-1.pdf}

Finally, using the \texttt{table} function provides an easy way of
checking the number of oberservations per month, per feeding plot, and
per feeding type. This is useful to check that there seem to be a
reasonable number of observations to proceed with the analysis.

\begin{Shaded}
\begin{Highlighting}[]
\KeywordTok{table}\NormalTok{(OC}\OperatorTok{$}\NormalTok{Month)}
\end{Highlighting}
\end{Shaded}

\begin{verbatim}
## 
## Dec Jan 
##  79 118
\end{verbatim}

\begin{Shaded}
\begin{Highlighting}[]
\KeywordTok{table}\NormalTok{(OC}\OperatorTok{$}\NormalTok{FeedingPlot)}
\end{Highlighting}
\end{Shaded}

\begin{verbatim}
## 
##  A  B  C 
## 66 53 78
\end{verbatim}

\begin{Shaded}
\begin{Highlighting}[]
\KeywordTok{table}\NormalTok{(OC}\OperatorTok{$}\NormalTok{FeedingType)}
\end{Highlighting}
\end{Shaded}

\begin{verbatim}
## 
## Hammerers  Stabbers 
##       165        32
\end{verbatim}

\emph{Note}: as we will see later, we have actually made a major error
here. There is in reality a problem with the data, and so really we
should not have stopped exploring it now before going ahead with our
analyses. We will come back to this issue\ldots{}

\subsection{4. Applying a linear regression
model}\label{applying-a-linear-regression-model}

We're going to do a simple linear model with the \texttt{lm} command,
looking at all interactions. With so many different predictors in the
model it is sometimes useful to use the \texttt{drop1} command to check
that the higher-level 3-way interaction terms are significant:

\begin{Shaded}
\begin{Highlighting}[]
\NormalTok{M1 <-}\StringTok{ }\KeywordTok{lm}\NormalTok{(ShellLength }\OperatorTok{~}\StringTok{ }\NormalTok{FeedingType }\OperatorTok{*}\StringTok{ }\NormalTok{FeedingPlot }\OperatorTok{*}\StringTok{ }\NormalTok{Month,}
         \DataTypeTok{data =}\NormalTok{ OC)}
\KeywordTok{print}\NormalTok{(}\KeywordTok{summary}\NormalTok{(M1), }\DataTypeTok{digits =} \DecValTok{2}\NormalTok{)}
\end{Highlighting}
\end{Shaded}

\begin{verbatim}
## 
## Call:
## lm(formula = ShellLength ~ FeedingType * FeedingPlot * Month, 
##     data = OC)
## 
## Residuals:
##    Min     1Q Median     3Q    Max 
## -0.698 -0.192  0.009  0.178  1.392 
## 
## Coefficients:
##                                           Estimate Std. Error t value
## (Intercept)                                  2.208      0.076    29.0
## FeedingTypeStabbers                          0.632      0.234     2.7
## FeedingPlotB                                 0.124      0.113     1.1
## FeedingPlotC                                -0.197      0.098    -2.0
## MonthJan                                    -0.076      0.090    -0.8
## FeedingTypeStabbers:FeedingPlotB            -0.936      0.286    -3.3
## FeedingTypeStabbers:FeedingPlotC            -0.573      0.255    -2.2
## FeedingTypeStabbers:MonthJan                -0.987      0.286    -3.5
## FeedingPlotB:MonthJan                       -0.234      0.135    -1.7
## FeedingPlotC:MonthJan                        0.105      0.121     0.9
## FeedingTypeStabbers:FeedingPlotB:MonthJan    1.308      0.380     3.4
## FeedingTypeStabbers:FeedingPlotC:MonthJan    0.901      0.357     2.5
##                                           Pr(>|t|)    
## (Intercept)                                 <2e-16 ***
## FeedingTypeStabbers                          0.008 ** 
## FeedingPlotB                                 0.275    
## FeedingPlotC                                 0.045 *  
## MonthJan                                     0.401    
## FeedingTypeStabbers:FeedingPlotB             0.001 ** 
## FeedingTypeStabbers:FeedingPlotC             0.026 *  
## FeedingTypeStabbers:MonthJan                 7e-04 ***
## FeedingPlotB:MonthJan                        0.085 .  
## FeedingPlotC:MonthJan                        0.389    
## FeedingTypeStabbers:FeedingPlotB:MonthJan    7e-04 ***
## FeedingTypeStabbers:FeedingPlotC:MonthJan    0.013 *  
## ---
## Signif. codes:  0 '***' 0.001 '**' 0.01 '*' 0.05 '.' 0.1 ' ' 1
## 
## Residual standard error: 0.31 on 185 degrees of freedom
## Multiple R-squared:  0.15,   Adjusted R-squared:  0.094 
## F-statistic: 2.9 on 11 and 185 DF,  p-value: 0.0018
\end{verbatim}

\begin{Shaded}
\begin{Highlighting}[]
\KeywordTok{drop1}\NormalTok{(M1, }\DataTypeTok{test =} \StringTok{"F"}\NormalTok{)}
\end{Highlighting}
\end{Shaded}

\begin{verbatim}
## Single term deletions
## 
## Model:
## ShellLength ~ FeedingType * FeedingPlot * Month
##                               Df Sum of Sq    RSS     AIC F value   Pr(>F)
## <none>                                     18.170 -445.54                 
## FeedingType:FeedingPlot:Month  2    1.1962 19.366 -436.98  6.0896 0.002746
##                                 
## <none>                          
## FeedingType:FeedingPlot:Month **
## ---
## Signif. codes:  0 '***' 0.001 '**' 0.01 '*' 0.05 '.' 0.1 ' ' 1
\end{verbatim}

So, we can conclude that the 3-way term of Feeding Type x Feeding Plot x
Month is significant. Of course, being good ecological data scientists
you know by now that it is not sufficient just to look at the tabular
output of a linear model, but also want the graphical output. Issue the
command \texttt{plot(M1)} to check the overall model diagnostics; what
is your interpretation of the four standard plots, especially the first
(Residuals vs Fitted) and second (Normal Q-Q plot)??

Here you have three categorical predictors, so sometimes it is useful to
look at the residuals for each predictor separately. Here is the code
for the Feeding Type:

\begin{Shaded}
\begin{Highlighting}[]
\NormalTok{E1 <-}\StringTok{ }\KeywordTok{rstandard}\NormalTok{(M1) }\CommentTok{# Extract standardised residuals}
\NormalTok{p1r <-}\StringTok{ }\KeywordTok{ggplot}\NormalTok{(}\DataTypeTok{data=}\NormalTok{OC, }\KeywordTok{aes}\NormalTok{(}\DataTypeTok{x=}\NormalTok{FeedingType, }\DataTypeTok{y=}\NormalTok{E1)) }\OperatorTok{+}
\StringTok{  }\KeywordTok{geom_boxplot}\NormalTok{() }\OperatorTok{+}
\StringTok{  }\KeywordTok{geom_hline}\NormalTok{(}\KeywordTok{aes}\NormalTok{(}\DataTypeTok{yintercept=}\DecValTok{0}\NormalTok{), }\DataTypeTok{linetype=}\StringTok{"dashed"}\NormalTok{) }\OperatorTok{+}
\StringTok{  }\KeywordTok{theme_classic}\NormalTok{()}
\NormalTok{p1r}
\end{Highlighting}
\end{Shaded}

\includegraphics{oystercatchers_files/figure-latex/feeding type resids-1.pdf}

Modify the code for Feeding plot and month - do the residuals seem
roughly similar for each level of each factor, and centred around zero?
Use multiplot to but all three plots side-by-side.

\subsection{5. Model interpretation}\label{model-interpretation}

Given that there seem to be no major problems with our linear model, we
can now go ahead and interpret what is going on in more detail. Remember
that the value labelled


\end{document}
